\documentclass[a4paper,12pt]{article}
\usepackage[utf8]{inputenc}
\usepackage[polish]{babel}
\usepackage[T1]{fontenc}
\usepackage{geometry}
\usepackage{graphicx}
\usepackage{enumitem}
\usepackage{hyperref}
\usepackage{longtable}
\geometry{margin=2.5cm}

\title{Specyfikacja Projektu Zaliczenowego (Programowanie w Języku Java)}
\author{Radosław Ciepał \and Patryk Kowalczyk}
\textbf{https://github.com/KowalczykPatryk/Chatterly}
\date{\today}

\begin{document}

\maketitle

\section*{1. Tytuł Projektu}

\vspace{0.5 cm}
 \noindent
\textbf{Chatterly}

\section*{2. Opis i Cele Projektu}

\vspace{0.5 cm}
 \noindent

\textit{ Projekt ma na celu stworzenie aplikacji desktopowej, która umożliwia użytkownikom szyfrowaną komunikację. Główne cele projektu:}

\begin{itemize}
    \item Bezpieczna rejestracja i logowanie użytkowników.
    \item Tworzenie, wysyłanie i odbieranie wiadomości tekstowych w czasie rzeczywistym
    \item Wyświetlanie listy kontaktów / znajomych
    \item Powiadomienia o przychodzących wiadomościach
    \item Wyświetlanie statusu online/offline
    \item Obsługa wylogowania
\end{itemize}

\section*{3. Wymagania Funkcjonalne}

\vspace{0.5 cm}
 \noindent

\begin{itemize}
    \item Rejestracja, logowanie, zarządzanie sesją.
    \item Szyfrowanie wiadomości „end-to-end”
    \item Przechowywanie historii czatu w relacyjnej bazie danych na serwerze (zaszyfrowanej)
    \item Przechowywanie tablicy urzytkowników w relacyjnej bazie danych na serwerze
    \item Cache kilku ostatnich wiadomości oraz kontaktów w lokalnej relacyjnej bazie danych (zaszyfrowanej)
    \item Zarządzanie kluczami kryptograficznymi (generowanie par kluczy asymetyrcznych oraz klucza symetrycznego, wymiana publicznych kluczy)
    \item Bezpieczne zamykania sesji po upłynięciu timeout
\end{itemize}

\section*{4. Wymagania Niefunkcjonalne}

\vspace{0.5 cm}
 \noindent
\begin{itemize}
    \item \textbf{Wydajność:} Czas przesyłu danych poniżej sekundy.
    \item \textbf{Skalowalność:}
        \begin{itemize}
        \item Obsługiwanie wielu wątków na serwerze - obsługiwane domyślnie przez Ecplipse Tyrus
        \end{itemize}
    \item \textbf{Bezpieczeństwo:}
        \begin{itemize}
            \item W serwerowej bazie danych znajdują sie hashowane hasła (bcrypt)
            \item Szyfrowanie komunikacji TCP serwer-klient (rejestracja, logowanie, wymiana endpointów API) (TSL/SSL)
            \begin{itemize}
                \item Uzgodnienie wersji protokołu (TLS 1.2/1.3),
                \item Wymiana kluczy asymetrycznych (certyfikat + klucz prywatny serwera),
                \item Uzgodnienie klucza sesji (symetrycznego),
                \item Szyfrowanie ruchu symetrycznym algorytmem (np. AES-GCM).
            \end{itemize}
            \item Uwierzytelnić serwer za pomocą TLS/SSL certyfikat X.509 dla domeny serwera (Let’s Encrypt)(KeyStore)
            \item Szyfrowanie „end-to-end” wiadomości (hybrydowy schemat kryptograficzny) (KeyStore)
            \item Utrzymywanie Ping/Pong(Heartbeat) między klientem a serverem (WebSocket idle timeout)
            \item Weryfikacja klienta i kontrola czy ma uprawnienia do przesyłu (JWT (JSON Web Token))
        \end{itemize}
    \item \textbf{Użyteczność:} Interfejs zgodny z odpowiednikiem z web WCAG (AA).
\end{itemize}

\section*{5. Architektura Systemu}
\textit{Poniżej opisano ogólną architekturę systemu czatu, jego główne komponenty oraz zależności między nimi.}
\vspace{0.5 cm}
 \noindent
\textbf{Typ architektury: Klient-Serwer (MVC / warstwowa).}

\textbf{System składa się z dwóch głównych części:}
\textbf{klienta desktopowego} (JavaFX) oraz
\textbf{serwera} (Java + REST + WS). Komponenty te komunikują się ze sobą za pośrednictwem
\textbf{TLS/SSL}, a autoryzacja klienta odbywa się przy pomocy
\textbf{JWT}. Ponadto do wymiany wiadomości w czasie rzeczywistym wykorzystywany jest
\textbf{WebSocket}. Ecplipse Tyrus implementuje Java API for WebSocket. Działa na serwerze i zarządza handshakami, sesjami WebSocket,
cyklem życia połączenia, heartbeat, konfiguracją zabezpieczeń oraz endpointami. Jersey obsługuje REST API endpoints implementując JAX-RS.


\subsection*{REST (HTTPS)}
\begin{itemize}
    \item Służy do operacji CRUD i pojedynczych wywołań „na żądanie”:
    \begin{itemize}
        \item Rejestracja (POST /api/register)
        \item Logowanie (POST /api/login) → zwraca JWT
        \item Odświeżanie tokena (POST /api/refreshToken)
        \item Pobranie listy znajomych (GET /api/users/friends)
        \item Pobranie historii czatu (GET /api/messages?withUser={id} limit={N})
        \item Pobranie/aktualizacja klucza publicznego (GET/PUT /api/users/{id}/publicKey)
    \end{itemize}
    \item Każde żądanie REST zawiera w nagłówku Authorization: Bearer <JWT>, który został uzyskany po zalogowaniu.
    \item Połączenia REST są bezstanowe (każde żądanie musi mieć ważny JWT) i szyfrowane TLS-em (HTTPS).
\end{itemize}

\subsection*{WebSocket (WSS)}
\begin{itemize}
    \item Służy wyłącznie do wymiany wiadomości w czasie rzeczywistym (push/pull).
    \item To „trwałe” połączenie TCP nad TLS:
    \begin{itemize}
        \item Przy handshake (otwieraniu WSS) klient dołącza ten sam JWT w nagłówku Authorization.
        \item Serwer sprawdza podpis i datę wygaśnięcia tokena. Gdy JWT jest poprawny, sesja WS jest akceptowana.
    \end{itemize}
    \item Po otwarciu WebSocket klient wysyła/odbiera kolejne zaszyfrowane (E2E) pakiety JSON bez potrzeby ponownego dołączania JWT w każdej wiadomości.
\end{itemize}

\subsection*{Logika przepływu:}
\begin{enumerate}
    \item Klient wysyła REST-owe POST /api/login → dostaje JWT.
    \item Klient (WSS) otwiera wss://server/chat z nagłówkiem Authorization: Bearer <JWT>.
    \item Serwer w handshake weryfikuje JWT i przyznaje sesję WS.
    \item Od teraz klient wysyła szyfrowane E2E-wiadomości przez WebSocket, a serwer tylko je przekazuje (plus buforuje offline).
    \item Gdy JWT wygaśnie, przy próbie ponownego connectToServer() WebSocket handshake się nie uda → klient musi przez REST odświeżyć token, a potem ponowić WSS.
\end{enumerate}

\subsection*{Schemat:}
\begin{itemize}
    \item Frontend: JavaFX + CSS
    \item Backend: Java + Tyrus + Jersey + Grizzly
    \item Server: Render Hobby Plan
    \item Baza danych: PostgreSQL, SQLite, JDBC (Java Database Connectivity)
    \item WebSocket: WebSocket
    \item Autoryzacja: JWT, certyfikat X.509
    \item Hashowanie: bcrypt, SQLCipher
    \item Przechowywanie kluczy i certyfikatów: KeyStore
    \item Generowanie TLS/SSL certyfikat X.509: Let’s Encrypt
\end{itemize}



\section*{6. Zastosowane Algorytmy}

\vspace{0.5 cm}
 \noindent
\begin{itemize}
    \item \textbf{Algorytmy uwierzytelniania:}
    \begin{itemize}
        \item \emph{JWT:}
        \begin{itemize}
            \item   Algorytm sygnatury: HMAC-SHA256 (HS256) lub RSA-SHA256 (RS256) – w zależności od tego, czy używasz klucza symetrycznego (sekret) czy klucza prywatnego/publicznego.
            \item   Claims: \texttt{{ sub: userId, iat: issuedAt, exp: expiresAt, roles: [] }}.
            \item   Weryfikacja podpisu oraz pola \texttt{exp} przed przyznaniem dostępu.
        \end{itemize}
        \item \emph{Hashowanie haseł:}
        \begin{itemize}
            \item bcrypt z solą (np. 12 rund generowania soli, parametr \texttt{cost=12}).
            \item Przechowywanie w bazie tylko 60-znakowego ciągu zwierajacego sol i hash.
            \item Weryfikacja \texttt{BCrypt.checkpw(plainPassword, storedHash)}.
        \end{itemize}
    \end{itemize}
    \item \textbf{Zarządzanie sesją:}
    \begin{itemize}
        \item \emph{Access Token (JWT):}
        \begin{itemize}
            \item   Czas życia (TTL) typowo 15–60 minut (\texttt{exp = now + TTL\_minutes}).
            \item   Po wygaśnięciu odrzucany w każdej weryfikacji (\texttt{if now > exp → 401}).
        \end{itemize}
        \item \emph{Refresh Token:}
        \begin{itemize}
            \item   Dłuższy TTL (np. 7 dni), zapisany w bazie z flagą \texttt{revoked}.
            \item   Endpoint \texttt{POST /api/refreshToken}: weryfikacja podpisu i obecności w bazie → wydanie nowego Access Tokena (i opcjonalnie nowego Refresh Tokena, unieważnienie starego).
        \end{itemize}
        \item \emph{Middleware / Interceptor:}
        \begin{itemize}
            \item   Każde wywołanie REST i handshake WebSocket weryfikuje JWT; jeśli \texttt{exp} przekroczone → próba odświeżenia (jeśli nagłówek \texttt{Refresh-Token} jest prawidłowy), w przeciwnym razie zwraca 401 i wymaga ponownego logowania.
            \item   W WebSocket: przy otrzymaniu błędu „token expired” klient automatycznie próbuje odświeżyć token i wznowić połączenie WSS.
        \end{itemize}
    \end{itemize}
    \item \textbf{Szyfrowanie „end-to-end” wiadomości:}
    \begin{itemize}
        \item \emph{Asymetryczne (klucze długoterminowe):}
        \begin{itemize}
            \item   RSA 2048-bit z OAEP (RSA/ECB/OAEPWithSHA-256AndMGF1Padding)
            \begin{itemize}
                \item \textbf{RSA} Algorytm szyfrowania asymetrycznego, w którym każdy użytkownik ma parę kluczy.
                \item \textbf{ECB (Electronic Codebook)} etykieta trybu szyfrowania dla bloków danych
                \item \textbf{OAEP (Optimal Asymmetric Encryption Padding)} wypełniania klucz do stałej długości
                \item \textbf{SHA (Secure Hash Algorithm)} konkretny algorytm skrótu używany wewnątrz OAEP do generowania maski i mieszanki bajtów.
            \end{itemize}
            \item  \textbf{lub}  ECC na krzywej secp256r1 (ECDH + AES-GCM) – generacja kluczy za pomocą \texttt{KeyPairGenerator.getInstance("EC")}, krzywa \texttt{SECP256R1}.
            \begin{itemize}
                \item \textbf{ECC (Elliptic Curve Cryptography)} Szyfrowanie asymetryczne oparte na własnościach krzywych eliptycznych
                \item \textbf{Krzywa secp256r1} Jeden z powszechnie stosowanych parametrów ECC
                \item \textbf{ECDH (Elliptic Curve Diffie–Hellman)} Protokół wymiany klucza, który korzysta z ECC
                \item \textbf{AES-GCM (Advanced Encryption Standard – Galois/Counter Mode)} Algorytm symetryczny łączący szyfrowanie blokowe AES z trybem GCM
            \end{itemize}
            \item   Klucz publiczny A i B przechowywany w bazie; klucz prywatny użytkownika zaszyfrowany hasłem w lokalnym keystore.
        \end{itemize}
        \item \emph{Symetryczne (klucze sesyjne):}
        \begin{itemize}
            \item   AES-GCM 256-bit (\texttt{Cipher.getInstance("AES/GCM/NoPadding")}), generacja IV 12 bajtów losowych.
            \begin{itemize}
                \item \textbf{AES (Advanced Encryption Standard)} symetryczny algorytm szyfrowania blokowego
                \item \textbf{Szyfrowanie (Counter Mode):} blok danych jest liczony sekwencyjnie (nonce + licznik), zaszyfrowany AES, a wynik XORowany z tekstem jawnym.
                \item \textbf{Autentykacja (Galois Field):}  generuje tag (128-bitowy) HMAC-podobny, obliczony nad całym szyfrogramem i opcjonalnymi danymi dodatkowo uwierzytelnianymi (AAD(Additional Authenticated Data))
            \end{itemize}
            \item   Przy każdej wiadomości:
            \begin{enumerate}
                \item   \texttt{SecretKey sessionKey = KeyGenerator.getInstance("AES").init(256).generateKey();}
                \item   \texttt{Cipher aes = Cipher.getInstance("AES/GCM/NoPadding");}
                \item   \texttt{GCMParameterSpec spec = new GCMParameterSpec(128, iv);}
                \item   \texttt{ciphertext = aes.doFinal(plaintext);}
            \end{enumerate}
        \end{itemize}
        \item \emph{Hybrydowy schemat pakowania:}
        \begin{itemize}
            \item   \texttt{encryptedKey = RSA\_OAEP(publicKeyRecipient, sessionKey)} lub \texttt{ECDH→derive sharedKey→AES-kdf→encryptedKey}.
            \item   JSON-payload:
            \begin{verbatim}
        {
          "to":"userB",
          "encryptedKey":"Base64(...)",
          "iv":"Base64(...)",
          "ciphertext":"Base64(...)",
          "timestamp":163XXXYYYY
        }
            \end{verbatim}
            \item   Odbiorca:
            \begin{enumerate}
                \item   Odzyskuje \texttt{sessionKey = RSA\_OAEP\_decrypt(privateKeyB, Base64(encryptedKey))}.
                \item   Odszyfrowuje \texttt{plaintext = AES\_GCM\_decrypt(sessionKey, iv, ciphertext)}.
            \end{enumerate}
        \end{itemize}
        \item \emph{Podpisywanie (opcjonalnie) dla integralności dodatkowej:}
        \begin{itemize}
            \item \textbf{HMAC (Hash-based Message Authentication Code)} mechanizm, który pozwala jednocześnie zweryfikować integralność danych oraz autentyczność ich nadawcy
            \item HMAC-SHA256 nad całym JSON-em lub nad \texttt{ciphertext + iv} z kluczem pochodnym od shared secret (w ECDH).
            \item Weryfikacja HMAC przed odszyfrowaniem w celu wykrycia manipulacji.
        \end{itemize}
    \end{itemize}
\end{itemize}

\section*{7. Tablice baz danych:}

\textbf{users} (przechowuje dane o kontach użytkowników)
\begin{itemize}
    \item id SERIAL PRIMARY KEY
    \item username VARCHAR(50) UNIQUE NOT NULL
    \item passwordHash VARCHAR(60) NOT NULL  – bcrypt-hash
    \item publicKey TEXT NOT NULL  – klucz publiczny RSA/ECC w formacie Base64
    \item createdAt TIMESTAMP WITH TIME ZONE DEFAULT NOW()
    \item updatedAt TIMESTAMP WITH TIME ZONE DEFAULT NOW()
\end{itemize}

\textbf{friends} (lista relacji „znajomości” między użytkownikami)
\begin{itemize}
    \item id SERIAL PRIMARY KEY
    \item userId INTEGER NOT NULL  — REFERENCES users(id) ON DELETE CASCADE
    \item friendId INTEGER NOT NULL  — REFERENCES users(id) ON DELETE CASCADE
    \item status VARCHAR(10) NOT NULL  – np. pending, accepted, blocked
    \item createdAt TIMESTAMP WITH TIME ZONE DEFAULT NOW()
    \item UNIKALNY INDEX (userId, friendId)
\end{itemize}

\textbf{messages} (zapis wszystkich wysłanych paczek – także tych już dostarczonych)
\begin{itemize}
    \item id BIGSERIAL PRIMARY KEY
    \item fromUser INTEGER NOT NULL  — REFERENCES users(id) ON DELETE CASCADE
    \item toUser INTEGER NOT NULL  — REFERENCES users(id) ON DELETE CASCADE
    \item iv BYTEA NOT NULL  – wektor inicjalizacyjny AES-GCM (12 B)
    \item encryptedKey BYTEA NOT NULL  – klucz AES zaszyfrowany RSA/OAEP (lub ECDH→KDF)
    \item ciphertext BYTEA NOT NULL  – zaszyfrowana treść wiadomości (AES-GCM)
    \item timestamp TIMESTAMP WITH TIME ZONE DEFAULT NOW()
    \item delivered BOOLEAN NOT NULL DEFAULT FALSE  – czy zostało „wysłane” do odbiorcy
\end{itemize}

\textbf{refreshTokens} (przechowuje refresh tokeny do odświeżania JWT)
\begin{itemize}
    \item id SERIAL PRIMARY KEY
    \item token VARCHAR(255) UNIQUE NOT NULL  – przechowujemy cały token (np. Base64 JWT)
    \item userId INTEGER NOT NULL  — REFERENCES users(id) ON DELETE CASCADE
    \item expiresAt TIMESTAMP WITH TIME ZONE NOT NULL
    \item revoked BOOLEAN NOT NULL DEFAULT FALSE
\end{itemize}

\textbf{userStatus} (szybki podgląd online/offline)
\begin{itemize}
    \item userId INTEGER PRIMARY KEY — REFERENCES users(id) ON DELETE CASCADE
    \item isOnline BOOLEAN NOT NULL DEFAULT FALSE
    \item lastSeen TIMESTAMP WITH TIME ZONE DEFAULT NOW()
\end{itemize}



\section*{8. Struktura Klas}

\vspace{0.5 cm}
 \noindent
\vspace{0.5 cm}
 \noindent
\subsection*{Model}
\begin{itemize}
    \item \texttt{User} – username, password, publicKey.
    \item \texttt{Message} – fromUser, toUser, iv, encryptedKey, ciphertext.
\end{itemize}

\subsection*{Serwisy}
\begin{itemize}
    \item \texttt{UserService} – rejestracja, logowanie, validateAccessToken, refreshTokens.
    \item \texttt{MessageService} – logika zarządzania wiadomościami.
    \item \item \texttt{FriendService} – logika zarządzania wiadomościami.
    \item \texttt{NotificationService} – generowanie powiadomień nowych wiadomości.
    \item \texttt{AuthService} – JWT, walidacja, bezpieczeństwo.
\end{itemize}

\subsection*{Kontrolery}
\begin{itemize}
    \item \texttt{UserController} – REST API użytkownika.
    \item \item \texttt{FriendController} – REST API zarządzania znajomymi.
    \item \texttt{MessageController} – REST API pobierania wiadomości.
\end{itemize}

\subsection*{Dodatkowe komponenty}
\begin{itemize}
    \item \texttt{WebSocketHandler} – wymiana wiadomości w czasie rzyczywistym.
    \item \texttt{DTO} – obiekty wymiany danych (Data Transfer Object).
    \item \texttt{DAO} – zapewnia abstrakcyjny interface do bazy danych (Data Access Object).
\end{itemize}

\section*{9. API}
\textit{Punkty końcowe API.}
\vspace{0.5 cm}
 \noindent
\subsection*{Endpointy:}
\begin{verbatim}
POST /api/users/register
POST /api/users/login
POST /api/users/refreshToken
POST /api/users/logout
GET /api/messages
\end{verbatim}

\section*{10. Interfejs Użytkownika}
\textit{Opis interfejsu graficznego aplikacji, jego struktury, logiki działania oraz dostępnych ekranów.}

\vspace{0.5 cm}
 \noindent
\subsection*{Główne widoki:}
\begin{itemize}
    \item \textbf{Ekran logowania i rejestracji:} Formularze z walidacją danych.
    \item \textbf{Pulpit użytkownika:}
    \begin{itemize}
        \item  Lista kontaktów z powiadomieniem o nowej wiadomości
        \item  Wyszukiwanie nowych użytkowników
        \item  Widok ostatnich wiadomości
        \item  Pole wpisywania i wysyłania wiadomości
    \end{itemize}
\end{itemize}

\section*{11. Etapy Realizacji / Harmonogram}
\textit{Plan wdrożenia projektu z podziałem na fazy i szacowanym czasem realizacji.}

\vspace{0.5 cm}
 \noindent
\begin{longtable}{|c|p{8cm}|c|}
\hline
\textbf{Faza} & \textbf{Opis} & \textbf{Czas} \\
\hline
Planowanie & Określenie wymagań, wybór technologii & 1 tydzień \\
\hline
Backend & Budowa API, modele danych & 1 tydzień \\
\hline
Frontend & Interfejs, integracja z API & 1 tydzień \\
\hline
\end{longtable}

\section*{12. Testowanie}
\textit{Sposoby testowania aplikacji, typy testów oraz używane narzędzia.}

\vspace{0.5 cm}
 \noindent
\begin{itemize}
    \item Testy jednostkowe (logika backendu i frontend).
    \item Testy integracyjne (API) JerseyTest, JUnit.
    \item Testy obciążeniowe – np. JMeter, Gatling.
\end{itemize}
\end{document}

